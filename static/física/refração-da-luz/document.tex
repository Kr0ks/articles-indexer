
\documentclass[
	12pt,				
	openright,			
	oneside,			
	a4paper,			
	brazil				
	]{abntex2}


% --------- PACOTES ---------
\usepackage{centernot}
\usepackage{amsmath}
\usepackage{amssymb}
\usepackage{lmodern}
\usepackage[T1]{fontenc}
\usepackage[utf8]{inputenc}
\usepackage{lastpage}
\usepackage{indentfirst}
\usepackage{graphicx}
\usepackage{microtype}

% --------- CAPA ---------
\titulo{Refração da luz}
\autor{Giuliano}
\local{Recife, PE}
\data{2023}

% --------- INFOS DO PDF ---------
\makeatletter
\hypersetup{
		pdftitle={\@title}, 
		pdfauthor={\@author},
    	pdfsubject={\imprimirpreambulo},
	    pdfcreator={LaTeX},
		pdfkeywords={abnt}{latex}{abntex}{abntex2}, 
		colorlinks=true,       		
    	linkcolor=blue,          	
    	citecolor=blue,        		
    	filecolor=magenta,
		urlcolor=blue,
}
\makeatother


\setlength{\parindent}{1.3cm}
\setlength{\parskip}{0.2cm}

\makeindex

\graphicspath{{./imgs}}
\begin{document}

\frenchspacing
\imprimircapa
\tableofcontents
\newpage

% --------- SEÇÕES ---------

\section[Introdução à refração da luz]{Introdução à refração da luz}
\textbf{A refração da luz significa que a velocidade da luz foi alterada devido à mudança no meio de propagação.} 
Nela, tem-se três grandezas: 
$n$, o qual significa índice de refração; 
$C$, o qual significa a constante da velocidade da luz no vácuo; 
e $V$, a qual diz respeito à velocidade da luz no meio.
Quanto maior o índice de refração, menor é a velocidade em que a luz se propagará nesse meio.
Há também, uma fórmula para o cálculo daquele:

$$
n = \frac{C}{V}
$$

\section[Índice de refração relativo]{Índice de refração relativo}
O índice de refração relativo é a razão entre dois índices de refrações absolutos; ou seja:

$$
n_{1,2} = \frac{n_1}{n_2} \implies n_{1,2} = \frac{\frac{C_1}{V_1}}{\frac{C_2}{V_2}} = \frac{\centernot{C_1}}{V_1}\cdot\frac{V_2}{\centernot{C_2}}
\therefore n_{1,2} = \frac{V_2}{V_1}
$$

\section[Desvio ângular]{Desvio ângular}
É possível calcular o desvio ângular através da equação $n_1 sen\theta_1 = n_2 sen\theta_2$, pois a razão entre o seno dos ângulos é igual à razão
entre as velocidades da luz nos meios incidentes.
\end{document}